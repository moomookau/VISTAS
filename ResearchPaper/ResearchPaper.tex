\documentclass{acm_proc_article-sp}
\usepackage[utf8]{inputenc}

\renewcommand{\paragraph}[1]{\vskip 6pt\noindent\textbf{#1 }}
\usepackage{hyperref}
\usepackage{graphicx}
\usepackage{url}

\providecommand{\tightlist}{%
  \setlength{\itemsep}{0pt}\setlength{\parskip}{0pt}}

\title{VISTAS - Visualising Industry Skill TAlent Shifts}


% Add imagehandling

\numberofauthors{2}
\author{
\alignauthor Cheryl Pay Wei Lin \\
        \affaddr{Singapore Management University}\\
       \email{\href{mailto:cheryl.pay.2019@mitb.smu.edu.sg}{\nolinkurl{cheryl.pay.2019@mitb.smu.edu.sg}}}
\and \alignauthor Chong Jia Jun Louis \\
        \affaddr{Singapore Management University}\\
       \email{\href{mailto:louis.chong.2019@mitb.smu.edu.sg}{\nolinkurl{louis.chong.2019@mitb.smu.edu.sg}}}
\and \alignauthor Lau Wei Han Amos \\
        \affaddr{Singapore Management University}\\
       \email{\href{mailto:amos.lau.2019@mitb.smu.edu.sg}{\nolinkurl{amos.lau.2019@mitb.smu.edu.sg}}}
\and }

\date{}

%Remove copyright shit
\permission{}
\conferenceinfo{} {}
\CopyrightYear{}
\crdata{}

% Pandoc syntax highlighting

% Pandoc citation processing
\newlength{\csllabelwidth}
\setlength{\csllabelwidth}{3em}
\newlength{\cslhangindent}
\setlength{\cslhangindent}{1.5em}
% for Pandoc 2.8 to 2.10.1
\newenvironment{cslreferences}%
  {}%
  {\par}
% For Pandoc 2.11+
\newenvironment{CSLReferences}[3] % #1 hanging-ident, #2 entry spacing
 {% don't indent paragraphs
  \setlength{\parindent}{0pt}
  % turn on hanging indent if param 1 is 1
  \ifodd #1 \everypar{\setlength{\hangindent}{\cslhangindent}}\ignorespaces\fi
  % set entry spacing
  \ifnum #2 > 0
  \setlength{\parskip}{#2\baselineskip}
  \fi
 }%
 {}
\usepackage{calc} % for calculating minipage widths
\newcommand{\CSLBlock}[1]{#1\hfill\break}
\newcommand{\CSLLeftMargin}[1]{\parbox[t]{\csllabelwidth}{#1}}
\newcommand{\CSLRightInline}[1]{\parbox[t]{\linewidth - \csllabelwidth}{#1}}
\newcommand{\CSLIndent}[1]{\hspace{\cslhangindent}#1}


\begin{document}
\maketitle

\begin{abstract}
{[}write abstract for research paper such as data viz methods, evals,
user interface design{]} VISTAS - Visualising Industry Skill TAlent
Shifts is a shiny app that aims to provide insights to talent, migration
and skill trends in an interactive and user-friendly way\ldots{} (no
more than 300 words)
\end{abstract}

\hypertarget{abstract}{%
\section{Abstract}\label{abstract}}

Numerous studies are conducted to analyse the impact of population
growth on GDP per capita. However, these studies usually tap onto
aggregated country level datasets and more can be done to deep dive into
the relationship between GDP per capita and employment growth or
migration in each industry. LinkedIn and World Bank Group have partnered
to release data from 2015 to 2019 on the employment growth and migration
for each industry and skill in various countries. By combining this data
with countries' GDP per capita in each year, individuals and countries
can better understand the rapidly evolving labour market and its impact
on countries' growth. We have designed and developed VISTAS (Visualising
Industry Skill TAlent Shifts), an interactive visual analytics
dashboard. Individuals can analyse the labour market, identify the
highly sought-after skills and countries that hold better employment
opportunities for each skill. Countries can study the impact of
employment growth and migration for each industry and skill on GDP per
capita and find out which key skills have been lost or gained.

\hypertarget{introduction}{%
\section{Introduction}\label{introduction}}

The LinkedIn and World Bank Group have partnered and released data from
2015 to 2019 that focuses on 100+ countries with at least 100,000
LinkedIn members each, distributed across 148 industries and 50,000
skill categories. This data aims to help government and researchers
understand rapidly evolving labor markets with detailed and dynamic
data.

Aside from being able to download the data in csv format, the
visualizations at \url{https://linkedindata.worldbank.org/data} are
limited in their variety and interactivity because the only allow one
parameter of choice. VISTA aims to be more dynamic and interactive as a
visual analytics dashboard created as a shiny app. Visualizations that
have been created are scatterplots, slope graphs, treemap and more.

Explain aim of VISTA, dynamic, interactive visual analytics dashboard

This paper reports on \ldots{} Section 1 (introduction), Section 2
(motivation/objectives), Section 3 (review of analytical techniques),
Section 4 (user interface design), Section 5 (application), Section 6
(insights), Section 7 (conclusion)

\hypertarget{motivation-and-objectives}{%
\section{Motivation and Objectives}\label{motivation-and-objectives}}

Through VISTA, we want to provide individuals and countries with
insights into various interest areas to benchmark themselves against the
global landscape. As an extension, we will be including macroeconomic
indicators of GDP growth from World Bank Organization to our data
visualisations. We envision that our project will help individual and
countries answer questions on employability, employment opportunities,
and migration and skill trends.

Instead of highlighting observations and presenting them directly, we
want to enable our users to discover the insights themselves. Hence our
objective is to create a simple tool for users to analyze the data in
various ways, including conducting statistical analyses of regression
and correlation. To do so we have added macroeconomic data from World
Bank to our dashboard.

The dashboard gives the user flexibility in selecting their own
variables for every visualization. One can choose the values they want
to see on their visualization, down to the x and y variables, and the
categorization e.g.~color by region.

\hypertarget{analytical-techniques}{%
\section{Analytical Techniques}\label{analytical-techniques}}

\emph{Limitations of existing visualizations}

Simple data visualisations are publicly available at
\url{https://linkedindata.worldbank.org/data}. But interactivity is
limited. Information is presented in one default way with no other
option and the user cannot conduct comparisons ``at one glance.''

Only one country is selected. \textless insert screenshot showing ``Pick
A Country''\textgreater{}

The map shows country migration when migration data for industry and
skills are also available. Aside from a point map, the output of
migration data can also be a choropleth map or slope graph.

Studies on macroeconomic data are usually done at country level and
there are limited studies to analyse the relationship between GDP per
capita growth rate and employment growth or migration in each industry.
These analyses are also shown as static charts in reports, lacking
interactivity.

Figure 1 shows the relationship between employment growth and interstate
migration (Tunny, G., 2015). Here, analysis is done at state level, but
not industry level. \textless insert Louis assignment Figure
1\textgreater{}

Figure 2 shows the relationship between GDP per capita and share of
employment in business services in 2000 for countries in Europe Kox, H.
\& Rubalcaba, L., 2007. However, this does not show how a change in
migration or employment growth in an industry will cause the change in
GDP per capita. \textless insert Louis assignment Figure 2\textgreater{}

\#\#Statistical Analyses

\hypertarget{talent-migration}{%
\subsection{Talent Migration}\label{talent-migration}}

The available visualisation only shows top 10 countries on the map, and
top 5 countries, industries and skills in the table. A user may want to
know more than just the top N for these metrics. The visualisation of
the map and table makes it difficult to visualise the inflow vs outflow
of the selected country. A visualisation which allows for comparison
between inflow and outflow could be useful to know if a country is
gaining or losing talent.

\hypertarget{industry-skills-needs}{%
\subsection{Industry Skills Needs}\label{industry-skills-needs}}

The interactive visualisation in \textbf{Industry Skills Needs panel}
allows users to pick an industry group and industry, but only the skills
valued in Year 2019 are shown. No option to change the year is provided.

One is hence unable to see the change in skills over the years in an
industry. By presenting the changes over the years, the user is able to
review trends and gather insights to the industry skills shift and
sought-after skills in a particular industry.

Describe analytical techniques used and explain how they can help
achieve objectives

Scatter plot: Visualise values of two variables i.e.~GDP per capita
growth, industry employment growth, industry migration, skill migration
Regression plot: Find out relationship between two variables, how much
industry employment growth, industry and skill migration impact GDP per
capita growth, to what extent does industry employment growth change
with industry and skill migration Correlation matrix: Find out strength
of relationship between each pair of variables,

\hypertarget{user-interface-design}{%
\section{User Interface Design}\label{user-interface-design}}

Explain each component of VISTA and how the combination of components
will provide users certain insights

\begin{itemize}
\tightlist
\item
  create tabs for various topics (compartmentalization)
\item
  ensuring positioning is not compromised with reduced window size /
  visualisations are not distorted when window width is reduced
\item
  action buttons are added to delay reactions and minimize lag -
  eventReactive()
\end{itemize}

\hypertarget{application}{%
\section{Application}\label{application}}

List and explain the packages used to create VISTA

\hypertarget{insights}{%
\section{Insights}\label{insights}}

What has the audience learned from your work? What new insights or
practices has your system enabled? A full blown user study is not
expected, but informal observations of use that help evaluate your
system are encouraged.

\hypertarget{conclusion}{%
\section{Conclusion}\label{conclusion}}

Future work - \ldots.

\hypertarget{acknowledgement}{%
\section{Acknowledgement}\label{acknowledgement}}

Prof Kam

\hypertarget{references}{%
\section*{References}\label{references}}
\addcontentsline{toc}{section}{References}

\hypertarget{refs}{}
\begin{CSLReferences}{0}{0}
\leavevmode\hypertarget{ref-fenner2012a}{}%
\CSLLeftMargin{{[}1{]} }
\CSLRightInline{Fenner, M. 2012. One-click science marketing.
\emph{Nature Materials}. 11, 4 (Mar. 2012), 261--263.}

\leavevmode\hypertarget{ref-meier2012}{}%
\CSLLeftMargin{{[}2{]} }
\CSLRightInline{Meier, R. 2012. \emph{Professinal Android 4 Application
Development}. John Wiley \& Sons, Inc.}

\end{CSLReferences}
\setlength{\parindent}{0in}

\end{document}
