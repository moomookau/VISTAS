\documentclass{acm_proc_article-sp}
\usepackage[utf8]{inputenc}

\renewcommand{\paragraph}[1]{\vskip 6pt\noindent\textbf{#1 }}
\usepackage{hyperref}
\usepackage{graphicx}
\usepackage{url}

\providecommand{\tightlist}{%
  \setlength{\itemsep}{0pt}\setlength{\parskip}{0pt}}

\title{VISTAS - Visualising Industry Skill TAlent Shifts}


% Add imagehandling

\numberofauthors{2}
\author{
\alignauthor Cheryl Pay Wei Lin \\
        \affaddr{Singapore Management University}\\
       \email{\href{mailto:cheryl.pay.2019@mitb.smu.edu.sg}{\nolinkurl{cheryl.pay.2019@mitb.smu.edu.sg}}}
\and \alignauthor Chong Jia Jun Louis \\
        \affaddr{Singapore Management University}\\
       \email{\href{mailto:louis.chong.2019@mitb.smu.edu.sg}{\nolinkurl{louis.chong.2019@mitb.smu.edu.sg}}}
\and \alignauthor Lau Wei Han Amos \\
        \affaddr{Singapore Management University}\\
       \email{\href{mailto:amos.lau.2019@mitb.smu.edu.sg}{\nolinkurl{amos.lau.2019@mitb.smu.edu.sg}}}
\and }

\date{}

%Remove copyright shit
\permission{}
\conferenceinfo{} {}
\CopyrightYear{}
\crdata{}

% Pandoc syntax highlighting

% Pandoc citation processing
\newlength{\csllabelwidth}
\setlength{\csllabelwidth}{3em}
\newlength{\cslhangindent}
\setlength{\cslhangindent}{1.5em}
% for Pandoc 2.8 to 2.10.1
\newenvironment{cslreferences}%
  {}%
  {\par}
% For Pandoc 2.11+
\newenvironment{CSLReferences}[3] % #1 hanging-ident, #2 entry spacing
 {% don't indent paragraphs
  \setlength{\parindent}{0pt}
  % turn on hanging indent if param 1 is 1
  \ifodd #1 \everypar{\setlength{\hangindent}{\cslhangindent}}\ignorespaces\fi
  % set entry spacing
  \ifnum #2 > 0
  \setlength{\parskip}{#2\baselineskip}
  \fi
 }%
 {}
\usepackage{calc} % for calculating minipage widths
\newcommand{\CSLBlock}[1]{#1\hfill\break}
\newcommand{\CSLLeftMargin}[1]{\parbox[t]{\csllabelwidth}{#1}}
\newcommand{\CSLRightInline}[1]{\parbox[t]{\linewidth - \csllabelwidth}{#1}}
\newcommand{\CSLIndent}[1]{\hspace{\cslhangindent}#1}


\begin{document}
\maketitle

\begin{abstract}
{[}write abstract for research paper such as data viz methods, evals,
user interface design{]} VISTAS - Visualising Industry Skill TAlent
Shifts is a shiny app that aims to provide insights to talent, migration
and skill trends in an interactive and user-friendly way\ldots{} (no
more than 300 words)
\end{abstract}

\hypertarget{introduction}{%
\section{Introduction}\label{introduction}}

The LinkedIn and World Bank Group have partnered and released data from
2015 to 2019 that focuses on 100+ countries with at least 100,000
LinkedIn members each, distributed across 148 industries and 50,000
skill categories. This data aims to help government and researchers
understand rapidly evolving labor markets with detailed and dynamic
data.

Through our project, we want to provide individuals and countries with
insights into various interest areas to benchmark themselves against the
global landscape. As an extension, we will be including macroeconomic
indicators of GDP growth from World Bank Organization to our data
visualisations. We envision that our project will help individual and
countries answer questions on employability, employment opportunities,
and migration and skill trends.

\hypertarget{evaluation}{%
\section{Evaluation}\label{evaluation}}

Simple data visualisations are publicly available at
\url{https://linkedindata.worldbank.org/data}.

\hypertarget{talent-migration}{%
\subsubsection{Talent Migration}\label{talent-migration}}

The available visualisation only shows top 10 countries on the map, and
top 5 countries, industries and skills in the table. A user may want to
know more than just the top N for these metrics. The visualisation of
the map and table makes it difficult to visualise the inflow vs outflow
of the selected country. A visualisation which allows for comparison
between inflow and outflow could be useful to know if a country is
gaining or losing talent.

\hypertarget{industry-skills-needs}{%
\subsubsection{Industry Skills Needs}\label{industry-skills-needs}}

The interactive visualisation in \textbf{Industry Skills Needs panel}
allows users to pick an industry group and industry, but only the skills
valued in Year 2019 are shown. No option to change the year is provided.

One is hence unable to see the change in skills over the years in an
industry. By presenting the changes over the years, the user is able to
review trends and gather insights to the industry skills shift and
sought-after skills in a particular industry.

\hypertarget{user-interface-design}{%
\section{User Interface Design}\label{user-interface-design}}

\begin{itemize}
\tightlist
\item
  create tabs for various topics (compartmentalization)
\item
  ensuring positioning is not compromised with reduced window size /
  visualisations are not distorted when window width is reduced
\end{itemize}

\hypertarget{theme}{%
\subsection{Theme}\label{theme}}

A drop-down list is provided for users to choose their theme and
preferred colour of the app.

\hypertarget{industry-skills-needs-1}{%
\subsubsection{Industry Skills Needs}\label{industry-skills-needs-1}}

To illustrate changes in industry skill needs over the years, a barplot
is created and complemented with a data table. A drop-down list is
provided for users to choose their

Duis nec purus sed neque porttitor tincidunt vitae quis augue. Donec
porttitor aliquam ante, nec convallis nisl ornare eu. Morbi ut purus et
justo commodo dignissim et nec nisl. Donec imperdiet tellus dolor, vel
dignissim risus venenatis eu. Aliquam tempor imperdiet massa, nec
fermentum tellus sollicitudin vulputate. Integer posuere porttitor
pharetra. Praesent vehicula elementum diam a suscipit. Morbi viverra
velit eget placerat pellentesque. Nunc congue augue non nisi ultrices
tempor.

\hypertarget{insights}{%
\section{Insights}\label{insights}}

What has the audience learned from your work? What new insights or
practices has your system enabled? A full blown user study is not
expected, but informal observations of use that help evaluate your
system are encouraged.

Duis nec purus sed neque porttitor tincidunt vitae quis augue. Donec
porttitor aliquam ante, nec convallis nisl ornare eu. Morbi ut purus et
justo commodo dignissim et nec nisl. Donec imperdiet tellus dolor, vel
digni.

\hypertarget{conclusion}{%
\section{Conclusion}\label{conclusion}}

Future work - \ldots. Duis nec purus sed neque porttitor tincidunt vitae
quis augue. Donec porttitor aliquam ante, nec convallis nisl ornare eu.
Morbi ut purus et justo commodo dignissim et nec nisl. Donec imperdiet
tellus dolor, vel dignissim risus venenatis eu. Aliquam tempor imperdiet
massa, nec fermentum tellus sollicitudin vulputate. Integer posuere
porttitor pharetra. Praesent vehicula elementum diam a suscipit. Morbi
viverra velit eget placerat pellentesque. Nunc congue augue non nisi
ultrices tempor.

\hypertarget{references}{%
\section*{References}\label{references}}
\addcontentsline{toc}{section}{References}

\hypertarget{refs}{}
\begin{CSLReferences}{0}{0}
\leavevmode\hypertarget{ref-fenner2012a}{}%
\CSLLeftMargin{{[}1{]} }
\CSLRightInline{Fenner, M. 2012. One-click science marketing.
\emph{Nature Materials}. 11, 4 (Mar. 2012), 261--263.}

\leavevmode\hypertarget{ref-meier2012}{}%
\CSLLeftMargin{{[}2{]} }
\CSLRightInline{Meier, R. 2012. \emph{Professinal Android 4 Application
Development}. John Wiley \& Sons, Inc.}

\end{CSLReferences}
\setlength{\parindent}{0in}

\end{document}
